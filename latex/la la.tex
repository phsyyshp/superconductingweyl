

\documentclass[]{article}

%opening
\title{Calculation of conductance in a system with a single Weyl node}
\author{Oguz Turker}

\begin{document}

\maketitle

\begin{abstract}

\end{abstract}

\section{Commutation rules}
The fermionic operators that creates(anhilliates) particles in a given band can be written in terms of slave operators as;
\begin{equation}
a_{k_{\pm}}=f_{k_{\pm}}\tau_k^x.
\end{equation}
Operators $a$ and $f$ obey usual fermionic anti-commutation relations which are given as,
 \begin{equation}
\{f_k,f^\dagger_{k+q}\}=\delta_{0,q}, \label{eq:fcom}
 \end{equation} 
 and,
 \begin{equation}
 \{a_k,a^\dagger_{k+q}\}=\delta_{0,q}\label{eq: acom},
  \end{equation} 
  if we substitute eq \ref{eq:fcom} to eq \ref{eq: acom} we have 
  \begin{equation}
  \tau_{k}f_{k+q}f^\dagger_{k}\tau_{k+q}-f_{k+q}\tau_{k+q}  \tau_{k}f^\dagger_{k}=-\delta_{0,q}+\tau_k\delta_{0,q}\tau_{k+q}
   \end{equation} 
   \section{Current operator}
   Since we are working in momentum space we will write the current operator and the continuity equation in momentum space. The Fourier transform of the current operator  is;
   \begin{equation}
   J(r)=\frac{1}{V}\int dq J(q)e^{iqr} 
   \end{equation}
   and the Fourier transform of the density operator is 
   \begin{equation}
      \rho(r)=\frac{1}{V}\sum_{q} dq \rho(q)e^{iqr},
   \end{equation}
   where 
   \begin{equation}
   \rho(q)=\sum_{k}a^{\dagger}_k a_{k+q} \label{eq:dq}.
   \end{equation}
   We can find the continuity equation for a given $q$ as 
   \begin{equation}
      \dot{\rho}(q)=-iqJ(q) \label{eq: ceq}
    \end{equation}
    if we use  Heisenberg's equation of motion, we have
    \begin{equation}
    [H,\rho_q](t)=q J_q.
    \end{equation}
Now let us  write   eq \ref{eq:dq} in terms of slave operators 
    \begin{equation}
    \rho_q=\sum_{k}\tau_{k}f^\dagger_{k}f_{k+q}\tau_{k+q}.
    \end{equation}
    Our hypothesis is current is carried by only $f$ particle. We want to show that the current operator consists of only $f$ and $f^\dagger$ operators.
     If we use eq\ref{eq:fcom} we find that 
         \begin{equation}
         \rho_0=\sum_{k}f^\dagger_{k}f_{k}
         \end{equation}
         but, $\rho_0$ does not contribute to the current which is clearly seen  by eq\ref{eq: ceq}. The $\rho_{q\neq0}$ cannot be written only in terms of  $f$ operators. For now it seems to be that this slave spin approach does not make the problem easier, nor the $J$ can be written in terms of solely $f$'s.
    \end{document}
    
